
\documentclass[a4paper]{report}

\usepackage[italian]{babel}
\usepackage[utf8]{inputenc}
\usepackage{amsmath}
\usepackage{graphicx}

\pagestyle{plain}

\title{Pumping Lemma}

\author{Marco Moschettini}

\begin{document}
\maketitle

\chapter{Teorema}
\section{Ipotesi}
In un linguaggio \textit{infinito}, ogni stringa sufficientemente lunga deve avere una parte che si ripete.

\section{Pumping lemma per grammatiche context-free}
Se \textbf{L} è un linguaggio context-free, esiste un interno N tale che, per ogni stringa z di lunghezza almeno pari a N:
\begin{itemize}
\item z può essere riscritta come: \( z = uvwxy \quad con \left | z \right | \ge N \)
\item la parte centrale \(vwx\) ha lunghezza limitata: \(\left | vwx \right | \le N\)
\item v e x non sono nulle: \( \left | vx \right | \ge 1\)
\item \emph{tutte le stringhe della forma \(uv^iwx^iy \in L\)}
\end{itemize}
In pratica il lemma afferma che le due sottostringhe \(v\) e \(x\) possono essere "pompate" quanto si vuole ottenendo sempre stringhe di L

\section{Pumping lemma per linguaggi regolari}
Se L è un linguaggio regolare, esiste un intero M tale che, per ogni stringa z di lunghezza almeno pari a M:
\begin{itemize}
\item z può essere riscritta come: \(z = xyw \quad con \left | z \right | \geq M\)
\item la parte centrale \(xy\) ha lunghezza limitata: \( \left | xy \right | \leq M\)
\item \(y\) non è nulla: \(\left | y \right | \ge 1\)
\item \emph{tutte le stringhe della forma \(xy^iw \in L\)}
\end{itemize}
In pratica, qui il lemma afferma che la sottostringa \(y\) può essere \emph{pompata} quanto si vuole ottenendo sempre stringhe di L 


\chapter{Esempi}
\section{Esempio 1}
Prendiamo in esempio un linguaggio:
\[ L = \{a^nb^nc^n,\ n > 0 \}\]
dimostriamo che il linguaggio \textbf{non} è context-free:
\begin{itemize}
\item se L fosse context-free, esisterebbe un intero N che soddisferebbe il pumping lemma; consideriamo allora la stringa \(z = a^nb^nc^n\) prendiamo ad esempio 
\item scomponiamo z nei cinque pezzi uvwxy con \(\left | \textbf{vwx} \right | \le N\)
\item ad esempio prendiamo \(n = 6 \rightarrow \) \textbf{z} = ``aaaaaabbbbbbccccc''
\item prendiamo una sottostringa \textbf{vwx} di \textbf{z} al più lunga N (nel nostro caso scegliamo \(N = 5\)) = ``abbbb''.
\item in questa stringa cosa corrisponde a v/w/x? Ci sono più possibilità:
\begin{itemize}
\item \textbf{v} = a, \textbf{w} = bbbb, \textbf{x} = \emph{vuota}
\item \textbf{v} = \emph{vuota}, \textbf{w} = abbb, \textbf{x} = b
\item ecc\dots
\end{itemize}
\item tra le stringhe del linguaggio, della forma \(uv^iwx^iy\) ci sono anche quelle per cui \(i = 0\) ossia in cui \textbf{v} e \textbf{x} mancano. ossia dato che \textbf{u} e \textbf{y} sono quelle scelte da noi poco fa (\textbf{u} =``aaaaa'', \textbf{vwx} = ``abbbb'', \textbf{y} = ``bbcccccc'') e che il pezzo centrale \textbf{w} può essere:
\begin{itemize}
\item ``bbbb''
\item ``abbb''
\item ``bbb''
\item ``abb''
\end{itemize}
\item la stringa \textbf{uwy} (ottenuta tagliando \textbf{v} e \textbf{x}) risulta ``aaaaa'' + w + `` bbcccccc'', ovvero tutte le 6 'c' previste in fondo, ma meno "a" e/o meno "b" del necessario, perchè alcune sono state mangiate dalla sotto-stringa \textbf{vx}.
\item di conseguenza la stringa \textbf{uwy} \textbf{non appartiene al linguaggio} violando l'ipotesi! Di conseguenza il linguaggio L \textbf{non è context free}
\end{itemize}

\section{Esempio 2}
Prendiamo in esempio un linguaggio:
\[ L = \{a^p, \ \text{p \textbf{primo}}\}\]
dimostriamo che il linguaggio \textbf{non} è regolare:
\begin{itemize}
\item se L fosse regolare, esisterebbe un M che soddisferebbe il pumping lemma;
\item sia P un primo \(\ge\) M + 2 (sappiamo che esiste perchè i numeri primi sono infiniti): consideriamo allora la stringa \(z = a^p\):
\item scomponiamo z nei tre pezzi \textbf{xyw}, con \(\left | y \right | = r\); ne segue che \(\left | xw \right | = p - r\)
\item in base al lemma, se L fosse regolare, la nuova stringa \(xy^{p-r}w\) dovrebbe anch'essa appartenere al linguaggio.
\item peccato però che la lunghezza di tale stringa sia: \[\left | xp^{p-r}w \right | = \left | xw \right | + (p-r)\left |y\right | = (p-r) + (p-r)\left |y \right | = (p-r)(1+\left |y \right |) = \textbf{(p-r)(1-r)}\] ovvero \emph{non un numero primo}.
\item pertanto possiamo affermare che essa non appartiene al linguaggio e dunque \textbf{esso non è regolare}.
\end{itemize}

\section{Esempio 3}
Prendiamo in esempio un linguaggio:
\[ L = \{0^n1^n+k\ \text{con}\ n,k>0\}\]
dimostriamo che il linguaggio \textbf{non è regolare}.
\begin{itemize}
\item prendiamo n=2, k=2 da cui z = ``001111''
\item scomponiamo in \textbf{xyw}: \textbf{x} = 0, \textbf{y} = 01, \textbf{w} = 111
\item si deve avere che \(\forall i \in N,\ xy^iw \in L\)
\item prendendo i = 2, abbiamo che \(xy^2w = 00101111 \notin L\)
\item pertanto L \textbf{non è regolare}.
\end{itemize}
Dimostriamo ora che L è di tipo 2 trovando una grammatica che lo genera:
\(S \rightarrow 0S1\ | 01G \\ G \rightarrow 1G \ |\ 1 \) 
Il linguaggio è \textbf{context-free}. \newline\newline
\(S \rightarrow aSa\ |\ X \\ X \rightarrow aX\ |\ bX\ |\ a\ |\ b \) 

\end{document}